\documentclass[12pt,letterpaper]{article}

% - default packages

% - \usepackage[backend=biber,style=apa]{biblatex}
\usepackage[doublespacing]{setspace}
\usepackage{indentfirst}
\usepackage{parskip}

% - \addbibresource{../bibliography.bib}

\title{Research Question}
\author{Crystal Mandal}
\date{}

\begin{document}

\maketitle

\section*{Research Question:}
\textbf{How is the relationship between the image of Attica State Penitentiary 
and the voice of the incarcerated discoursed in contemporary (and specifically 
post-riot Music, letters, and essays?}

\section*{Justification}

This question analyses the musical discourse of Attica and the inmate. The 
inclusion of letters and essays may be controversial, but music is not exclusively 
that which produces sound. By including letters, we also bring into conversation 
the discourse of music as art outside of performance (are LaMonte Young's graphic 
scores music?) and redefine what can be analysed in a musical sense. This is 
important because Attica is the most important and resonant recent American 
prison riot, and the following prison abolition movements were taken extremely 
seriously at the time: though Attica has faded into history, we can listen to 
its echoes. 

% - \nocite{*}

% - \printbibliography

\end{document}
