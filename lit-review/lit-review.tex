\documentclass[12pt,letterpaper]{article}

% - default packages

\usepackage[backend=biber]{biblatex-chicago}
\usepackage[doublespacing]{setspace}
\usepackage{indentfirst}
\usepackage{parskip}

\addbibresource{../bibliography.bib}

\title{\vspace*{-72pt}Literature Review}
\author{Crystal Mandal}
\date{}

\begin{document}

\maketitle

As discussed in this course, the voice of an oppressed subject - physically, 
socially, aurally - is a rich subject for musicological discourse. Of note is 
that musicological discourse need not concern itself exclusively with a 
musical subject. In this project, I aim to clarify - or at least focus - an 
image of Attica State Penitentiary as a microcosm of the American Mass Incarceration 
system by examining music, essays, and letters from and about Attica Inmates around 
1969 to 1973 - contemporaneous with the Attica Riots. 

Some of my most important texts are: 

\begin{itemize}

	\item The Life of Paper, by Sharon Luk\\
	\fullcite{life-of-paper}
	
	\clearpage
	
	\item Coming Together, by Frederic Rzewski\\
	\fullcite{coming-together}

	\item Letters from Attica, by Samuel Melville\\
	\fullcite{letters-from-attica}
	
	\item Prisoner's Voices, by Daniel Metzer\\
	\fullcite{prisoners-voices}

\end{itemize}

The Life of Paper is, while incredibly dense, foundational in my 
understanding of letters as information-dense carriers for 
incarcerated/interned voices. Luk argues that the construction of 
the letter as message-over-distance and the asynchronicity of the 
messages (that is, the discrete and substantial amount of time 
between the sending of and receipt of the message), as well as the 
physical act of censorship and screening by prison officials, bestows 
the prison with an imagined, distant geography - that is, the usage 
of letter as principal communication distances the inmates from 
any neighbourhood and lumps incarcerated/interned voices together. 
Further, this usage of frameworks of imagined geographies links 
conversations about incarcerated/interned voices and critiques of 
Orientalist movements. Musicology is quite well prepared to discourse 
Orientalism in creative analysis - there is much literature of the 
music of Debussy and Ravel and Godowsky and Cage and their Oriental 
inspirations. Just as Luk analyses the form of the letter as a 
lexical prison, Metzer analyses the construction of Melville's 
letters and Rzewski's music - and puts the two into more explicit 
dialogue than Rzewski's own writing - to better formulate a musical 
understanding of incarceration. To Metzer, the structure of Rzewski's 
musics is fundamentally related to Rzewski's critiques of incarceration. 
These rhythmic patterns and structures can be used to analyse the structures 
of Melville's Lettes and Tisdale's Collected Poems in ``When the Smoke Cleared" 
(another source of mine).


\nocite{attica-poems}

\clearpage

\printbibliography

\end{document}
