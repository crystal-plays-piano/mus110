\documentclass[14pt, letterpaper]{article}
\usepackage[utf8]{inputenc}

% - default packages
% \usepackage[backend=biber, style=apa]{biblatex}
\usepackage[doublespacing]{setspace}
\usepackage{indentfirst}
\usepackage{parskip}
\usepackage{amsmath}
% \usepackage{changepage}
% \usepackage[dvipsnames]{xcolor}

% - set variables
% - \setlength{\parindent}{8ex}
% \definecolor{house-blue}{RGB}{0, 71, 187}



% - bibliography packages
% \usepackage[american]{babel}
%	\addto{\captionsenglish}{\renewcommand{\bibname}{Works that Inspired this Essay}}
% \usepackage{hyperref}
 \usepackage{csquotes}

% - import bib file
% \addbibresource{../bibliography.bib}

% - commands

% - title

\title{ \vspace*{-72pt} Topic Brainstorm }
\author{Crystal Mandal}
\date{}

\begin{document}

\maketitle

\section*{Politics of the American Folk Music Revival}
The American Folk Revival of the 1960's and 1970's attracted a lot of 
strong musical and political voices. Among the more interesting (to me) 
are Pete Seeger, Woody Guthrie, and, though not as well known or as 
explicitly involved, Frederic Rzewski. These musicians are all known 
to be very explicit in their political leanings/intentions. I'm interested 
in analysing the musical and lyrical content - as well as extramusical context - 
of their musics to better understand the politics of the American Folk Music 
revival. Some Research Questions are:
\begin{itemize}

	\item What common political notions are held by musicians involved 
	with the American Folk Music Revival?
	
	\item What about the ideological and commercial histories of the 
	American Folk Music Revival lend itself to the predominantly 
	progressive politics associated with the movement?
	
	\item What is said about the American Peoples/the American Musical Voice 
	by associating American Folk Music with leftist politics?
	
	\item How was the American Folk Music Revival shaped by the anti-war 
	sentiments of the 1960's?

\end{itemize}

\clearpage

\section*{Computer Music and Musical Voice/Intention}
Out of the rise of both computing power and availability of Computers during the 
20th century is born many types of uniquely computer-powered music. These musics 
all bring into question the role of the composer and performer in a modern 
technological context. Adjacent to these conversations is the critical discourse 
of AI produced musics. I'm interested in tracing the narrative of musical voice 
as it is transformed and reborn within various computer music paradigms. Some 
research questions are:

\begin{itemize}

	\item What is held common in the musics of Karlheinz 
	Stockhausen, Steve Reich, Daft Punk? 
	
	\item How is the Computer treated - historically and 
	contemporarily - as tool, collaborator, or composer? 
	
	\item Who is the producer when it comes to AI-generated 
	music? How does the AI-generation of music differ from other 
	computer-based productions of music?

\end{itemize}

\clearpage 

\section*{Musical Depictions of Incarceration... + Attica}
Much of Frederic Rzewski's music puts the inmate's voice at forefront. Compositions 
such as ``Coming Together" and ``Attica" concern themselves with particular 
depictions of the Attica State Prison Riots, whilst ``De Profundis" depicts 
Oscar Wilde's famous sentence served in Reading Jail, and ``It Makes a Long Time 
Man Feel Bad" from his North American Ballads relates incarceration and labour. 
More contemporary work ``If I Ruled the World" by NAS also brings Attica into 
cultural discourse, again putting Attica, carceral justice, reform, and abolition, 
and labour into the spotlight. Some research questions are:

\begin{itemize}

	\item How is Rzewski's music discoursing incarceration constructed, and 
	how does this depict incarceration?
	
	\item What is the relationship between prison and labour, and how can that 
	be depicted musically?
	
	\item Why do Attica, Soledad, Folsom, and more prisons, especially during 
	the 1960's and 1970's keep showing up in conversations about incarcerated 
	voices in America?
	
	\item How are musics of revolution and musics of incarceration intertwined?

\end{itemize}


\end{document} 
