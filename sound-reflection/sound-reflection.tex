\documentclass[14pt, letterpaper]{article}
\usepackage[utf8]{inputenc}

% - default packages
% \usepackage[backend=biber, style=apa]{biblatex}
\usepackage[doublespacing]{setspace}
\usepackage{indentfirst}
\usepackage{parskip}
\usepackage{amsmath}
% \usepackage{changepage}
% \usepackage[dvipsnames]{xcolor}

% - set variables
% - \setlength{\parindent}{8ex}
% \definecolor{house-blue}{RGB}{0, 71, 187}



% - bibliography packages
% \usepackage[american]{babel}
%	\addto{\captionsenglish}{\renewcommand{\bibname}{Works that Inspired this Essay}}
% \usepackage{hyperref}
 \usepackage{csquotes}

% - import bib file
% \addbibresource{../bibliography.bib}

% - commands

% - title

\title{ \vspace*{-72pt} Personal Sound Reflection }
\author{Crystal Mandal}
\date{}

\begin{document}

\maketitle

On the 26th of June, 2021, pianist-composer Frederic Rzewski passed away. 

Of course, when I heard the news, I was ambivalent. Who was this Rzewski? 
Sorry, I'm not a fan of ``contemporary music". It's too jarring and tasteless. 
And who even keeps track of contemporary composers? The great composers are 
great because they are dead. Surely no one alive could match them?

Then, on the 30th of June, youtube user ``Scriabin is my Dog" (I'm sure they 
had a less colourful username at the time, but the video I have saved is by 
``Scriabin is my Dog" currently) posted a score video of a piece titled ``Frederic Rzewski - The People United Will Never Be Defeated (Hamelin)". Now, of course, 
I loved Hamelin! I at least knew who he was, then, and all the great pianists 
had a name something-or-other I'd heard of before (nevermind the fact that 
I couldn't tell any of them apart or even admit to knowing confidently a single 
piece I liked recorded by Hamelin). And the thumbnail looked interesting: 
rather, it looked \textit{challenging}. Crazy, wide range arpeggios flooded 
the screen with an absurd number of accidentals. Surely this would be a fun thing 
to skim through and then never listen to again (like, persay, the Scriabin sonatas 
which I've since learned to love and truly appreciate). 

Thus begins, on a rather uninteresting and unimportant summer day, one of the 
most musically influential hours of my life. Rzewski's ``Thirty Six Variations 
on The People United Will Never Be Defeated" (commonly shortened to ``The People 
United") is one of the most immediately attractive and repulsive works I've ever 
encountered. Through much struggle and a fair share of truly grotesque music, the 
simple and triumphant theme is transformed from one of naïveté to one of complicated 
emotions and determination. 

I don't think I can properly describe the foundational musical effect this had 
on me. When I found the piece, I was going through a tough time musically. I was 
having difficulty placing my voice and choosing new repertoire, and I was 
beginning to lose enjoyment of listening to music. Everything sounded samey and 
a little boring and difficult to care about. It's really hard to have to pretend  
to like Wagner. My rather unconventional introduction to music of the 20th 
and 21st centuries and to the unique musical voice of Frederic Rzewski is, I 
believe, what makes me the pianist I am today. Just this past year, I performed 
his Four North American Ballads as part of a set in my Senior Recital, and I'm 
very glad I did. Through me, many more people around me have begun questioning 
the aesthetic qualities of music, and I hope to bring more ``unconventional", 
contemporary works to the stage.


\end{document} 
