\documentclass[12pt,letterpaper]{article}

% - default packages

\usepackage[backend=biber]{biblatex-chicago}
\usepackage[doublespacing]{setspace}
\usepackage{indentfirst}
\usepackage{parskip}
\usepackage{changepage}

\addbibresource{../bibliography.bib}

\title{\vspace*{-72pt}A Musical* Reconstruction of Attica image from Inmate-voiced 
Post-Riot Cultural Products`}
\author{Crystal Mandal}
\date{}

\begin{document}

\maketitle

\section*{Research Question:}
\textbf{How is the relationship between the image of Attica State Penitentiary 
and the voice of the incarcerated discoursed in contemporary (and specifically 
post-riot Music, letters, and essays?}

\section*{Rationale}

This question analyses the musical discourse of Attica and the inmate. The 
inclusion of letters and essays may be controversial, but music is not exclusively 
that which produces sound. By including letters, we also bring into conversation 
the discourse of music as art outside of performance (are LaMonte Young's graphic 
scores music?) and redefine what can be analysed in a musical sense. This is 
important because Attica is the most important and resonant recent American 
prison riot, and the following prison abolition movements were taken extremely 
seriously at the time: though Attica has faded into history, we can listen to 
its echoes. 

\fullcite{life-of-paper}

\section*{Bibliography}

\begin{adjustwidth}{2.5ex}{0pt}

	Sharon Luk's The Life of Paper is a remarkable - though painfully dense - 
	work narrating The Letter's role as voice and as communication line for 
	incarcerated peoples. Chapters Two (of Genealogy and Diaspora), Four 
	(of Censorship), Five (of Ephemera), and Six (of Profanity). The work's 
	complexity is just barely comprehensible, though it offers glimpses of 
	truly life-altering revelations. I plan to reconstruct my understanding of 
	her epistolary framework to apply to Jackson, Cleaver, Melville, and 
	Tisdale's Letters, Essays, and Poems to illustrate the prevalence of the 
	figure of the prison in their writing. By illustrating the figure of the 
	prison, I can then compare it with the depictions of prison in 
	contemporaneous pieces like Rzewski's Coming Together, as well as historic 
	pieces like Oscar Wilde's famous De Profundis. 

\end{adjustwidth}

\clearpage

\fullcite{letters-from-attica}

\begin{adjustwidth}{2.5em}{0pt}

	Letters from Attica is a collection of Letters, Essays, and 
	Newspaper articles written by political prisoner Samuel Melville 
	during his time served in various American prisons. The Collection 
	is preceded by a statement from Samuel's son, Joshua Melville, 
	who reflects on the difficulty of collecting and printing 
	these letters, as well as the Government censorship surrounding 
	the Attica Prison Riots. I plan to use some of Sam's letters as 
	creative ``grains" in constructing an image of the "indifferent 
	brutality" of pre-Attica-Riot living conditions for the incarcerated.
	Highlights include a particularly poetic and resonant letter from 
	16th May, 1970,  detailling Melville's experience at ``The Tombs" 
	(a nickname for Manhatton Detention Complex.), a report entitled 
	``An Anatomy of the Laundry", and snippets of a section Melville 
	published in the Attica Newsletter titled ``The Iced Pig". As a 
	related point, one of Melville's earlier letters is set to music 
	in Frederic Rzewski's ``Coming Together".

\end{adjustwidth}

\clearpage

\fullcite{prisoners-voices}

\begin{adjustwidth}{2ex}{0pt}

  This source is a unique perspective of a Musicologist. Here, Metzer 
  argues that the construction of Rzewski's music is itself discourse 
  on prison architecture, image, and abolition. Though Metzer indicates 
  that much of Rzewski's music is relevant to this conversation, he 
  focuses primarily on the works ``Coming Together" and ``Attica", which 
  are uniquely related in that they are both written about and in the 
  aftermath of the Attica State Prison Riots. He argues that the minimalist, 
  repeating musical structure of the pieces is representative of the mental 
  landscape of one in isolation, leading to a restless \textit{moto perpetuo}
  in ``Coming Together" and a still reflection in ``Attica". Metzer's work 
  is foundational in contextualising music as informationally dense and 
  a strong carrier of political messaging, especially in conversations 
  about carceral representations.
  
\end{adjustwidth}

\clearpage

\fullcite{coming-together}

\begin{adjustwidth}{2.5ex}{0pt}

  Frederic Rzewski was a very unique musical voice. Much of his music 
  (as emphasized in David Metzer's ``Prisoner's Voices") is politically 
  charged, contemporary, and specifically related to anti-war and 
  prison abolition movements. Coming Together specifically is a piece 
  written in the wake of the Attica Prison riot, with text from one 
  of Melville's letters. Rzewski noted that he was impressed by 
  ``poetic quality of the text and by its cryptic irony". Encoded 
  in the text and in the musical construction is the image of Attica: 
  a bleak, rigid structure that restrains a Revolutionary Black Soul. 
  ``Coming Together", along with partner piece ``Attica" functions as 
  a primary impetus for this project. They will both function as major 
  creative products from 1970's post-Riot in my synthesis. 

\end{adjustwidth}

\clearpage

\fullcite{attica-poems}

\begin{adjustwidth}{2.5ex}{0pt}

  ``When the Smoke Cleared" is a marvelous collection. This 
  collection of poems was written in the years following the 
  Attica Riots by inmates still residing at Attica State. The 
  collection was compiled from works written in editor Celes 
  Tisdale's poetry writing workshops. Tisdale served as teacher 
  at Attica from 1972 to 1975, where he ran three 16 week poetry 
  workshops, as well as other classes. Of note is the emotional 
  friction and contrast between the writings of visitor Tisdale 
  and his students, who all seem to be warmer, more direct, and 
  more violent (though Tisdale uses the language 	``unpolished") 
  in their approaches to poetry and writing than Tisdale. The 
  primary poems I wish to analyse and use in my cultural synthesis 
  are: 
  \begin{itemize}
  
    \item Poet - Raymond X. Webster
    
    \item Black Dolphin and Haiku - Harold E. Packwood  
    
    \item What Makes a Man Free? - Clarence Phillips
    
    \item The Cure - ``Jamail" Robert Simms
    
    \item 1st Page - Daniel Brown
    
    \item Remember This - Celes Tisdale
  
  \end{itemize}
  
\end{adjustwidth}

\section*{Literature Review}

s discussed in this course, the voice of an oppressed subject - physically, 
socially, aurally - is a rich subject for musicological discourse. Of note is 
that musicological discourse need not concern itself exclusively with a 
musical subject. In this project, I aim to clarify - or at least focus - an 
image of Attica State Penitentiary as a microcosm of the American Mass Incarceration 
system by examining music, essays, and letters from and about Attica Inmates around 
1969 to 1973 - contemporaneous with the Attica Riots. 

Some of my most important texts are: 

\begin{itemize}

	\item The Life of Paper, by Sharon Luk\\
	\fullcite{life-of-paper}
	
	\clearpage
	
	\item Coming Together, by Frederic Rzewski\\
	\fullcite{coming-together}

	\item Letters from Attica, by Samuel Melville\\
	\fullcite{letters-from-attica}
	
	\item Prisoner's Voices, by Daniel Metzer\\
	\fullcite{prisoners-voices}

\end{itemize}

The Life of Paper is, while incredibly dense, foundational in my 
understanding of letters as information-dense carriers for 
incarcerated/interned voices. Luk argues that the construction of 
the letter as message-over-distance and the asynchronicity of the 
messages (that is, the discrete and substantial amount of time 
between the sending of and receipt of the message), as well as the 
physical act of censorship and screening by prison officials, bestows 
the prison with an imagined, distant geography - that is, the usage 
of letter as principal communication distances the inmates from 
any neighbourhood and lumps incarcerated/interned voices together. 
Further, this usage of frameworks of imagined geographies links 
conversations about incarcerated/interned voices and critiques of 
Orientalist movements. Musicology is quite well prepared to discourse 
Orientalism in creative analysis - there is much literature of the 
music of Debussy and Ravel and Godowsky and Cage and their Oriental 
inspirations. Just as Luk analyses the form of the letter as a 
lexical prison, Metzer analyses the construction of Melville's 
letters and Rzewski's music - and puts the two into more explicit 
dialogue than Rzewski's own writing - to better formulate a musical 
understanding of incarceration. To Metzer, the structure of Rzewski's 
musics is fundamentally related to Rzewski's critiques of incarceration. 
These rhythmic patterns and structures can be used to analyse the structures 
of Melville's Lettes and Tisdale's Collected Poems in ``When the Smoke Cleared" 
(another source of mine).

\section*{Methodology}

This project aims to synthesize a new image of Attica ca. 1971 
(contemporary to the riots) to better understand a contemporary (ca. 2025) 
political climate surrounding mass incarceration. I plan to do this 
by analysing and deconstructing contemporaneous music, essays, and letters 
primarily written by Attica inmates. Because of this, my sources seem to 
be broadly non-musical: essays, articles, letters, poems, and very few 
scores or recordings. I believe by breaking down these creative works 
to their fundamental atoms/grains, we can synthesize a better, inmate-voiced 
image of Attica by connecting themes between these works. 



\nocite{*}

\printbibliography

\end{document}
