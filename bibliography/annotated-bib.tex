
\documentclass[14pt, letterpaper]{article}
\usepackage[utf8]{inputenc}

% - default packages
\usepackage[backend=biber, style=apa]{biblatex}
\usepackage[doublespacing]{setspace}
\usepackage{indentfirst}
\usepackage{parskip}
\usepackage{amsmath}
\usepackage{changepage}
% \usepackage[dvipsnames]{xcolor}

% - set variables
% - \setlength{\parindent}{8ex}
% \definecolor{house-blue}{RGB}{0, 71, 187}



% - bibliography packages
% \usepackage[american]{babel}
%	\addto{\captionsenglish}{\renewcommand{\bibname}{Works that Inspired this Essay}}
% \usepackage{hyperref}
 \usepackage{csquotes}

% - import bib file
\addbibresource{../bibliography.bib}

% - commands

% - title

\title{ \vspace*{-72pt} Literary Construction of Attica in Post-Riot Cultural Products }
\author{Crystal Mandal}
\date{}

\begin{document}

\maketitle


\fullcite{life-of-paper}

\begin{adjustwidth}{2.5ex}{0pt}

	Sharon Luk's The Life of Paper is a remarkable - though painfully dense - 
	work narrating The Letter's role as voice and as communication line for 
	incarcerated peoples. Chapters Two (of Genealogy and Diaspora), Four 
	(of Censorship), Five (of Ephemera), and Six (of Profanity). The work's 
	complexity is just barely comprehensible, though it offers glimpses of 
	truly life-altering revelations. I plan to reconstruct my understanding of 
	her epistolary framework to apply to Jackson, Cleaver, Melville, and 
	Tisdale's Letters, Essays, and Poems to illustrate the prevalence of the 
	figure of the prison in their writing. By illustrating the figure of the 
	prison, I can then compare it with the depictions of prison in 
	contemporaneous pieces like Rzewski's Coming Together, as well as historic 
	pieces like Oscar Wilde's famous De Profundis. 

\end{adjustwidth}

\clearpage

\fullcite{letters-from-attica}

\begin{adjustwidth}{2.5em}{0pt}

	Letters from Attica is a collection of Letters, Essays, and 
	Newspaper articles written by political prisoner Samuel Melville 
	during his time served in various American prisons. The Collection 
	is preceded by a statement from Samuel's son, Joshua Melville, 
	who reflects on the difficulty of collecting and printing 
	these letters, as well as the Government censorship surrounding 
	the Attica Prison Riots. I plan to use some of Sam's letters as 
	creative ``grains" in constructing an image of the "indifferent 
	brutality" of pre-Attica-Riot living conditions for the incarcerated.
	Highlights include a particularly poetic and resonant letter from 
	16th May, 1970,  detailling Melville's experience at ``The Tombs" 
	(a nickname for Manhatton Detention Complex.), a report entitled 
	``An Anatomy of the Laundry", and snippets of a section Melville 
	published in the Attica Newsletter titled ``The Iced Pig". 

\end{adjustwidth}

\clearpage

\fullcite{prisoners-voices}

\begin{adjustwidth}{2ex}{0pt}

  This source is a unique perspective of a Musicologist. Here, Metzer 
  argues that the construction of Rzewski's music is itself discourse 
  on prison architecture, image, and abolition. Though Metzer indicates 
  that much of Rzewski's music is relevant to this conversation, he 
  focuses primarily on the works ``Coming Together" and ``Attica", which 
  are uniquely related in that they are both written about and in the 
  aftermath of the Attica State Prison Riots. He argues that the minimalist, 
  repeating musical structure of the pieces is representative of the mental 
  landscape of one in isolation, leading to a restless \textit{moto perpetuo}
  in ``Coming Together" and a still reflection in ``Attica". Metzer's work 
  is foundational in contextualising music as informationally dense and 
  a strong carrier of political messaging. 
  
\end{adjustwidth}

\clearpage

\fullcite{coming-together}

\begin{adjustwidth}{2.5ex}{0pt}

  Frederic Rzewski was a very unique musical voice. Much of his music 
  (as emphasized in David Metzer's ``Prisoner's Voices") is politically 
  charged, contemporary, and specifically related to anti-war and 
  prison abolition movements. Coming Together specifically is a piece 
  written in the wake of the Attica Prison riot, with text from one 
  of Melville's letters. Rzewski noted that he was impressed by 
  ``poetic quality of the text and by its cryptic irony". Encoded 
  in the text and in the musical construction is the image of Attica: 
  a bleak, rigid structure that restrains a Revolutionary Black Soul. 
  ``Coming Together", along with partner piece ``Attica" functions as 
  a primary impetus for this project. They will both function as major 
  creative products from 1970's post-Riot in my synthesis. 

\end{adjustwidth}

\clearpage

\fullcite{attica-poems}

\begin{adjustwidth}{2.5ex}{0pt}

  ``When the Smoke Cleared" is a marvelous collection. This 
  collection of poems was written in the years following the 
  Attica Riots by inmates still residing at Attica State. The 
  collection was compiled from works written in editor Celes 
  Tisdale's poetry writing workshops. Tisdale served as teacher 
  at Attica from 1972 to 1975, where he ran three 16 week poetry 
  workshops, as well as other classes. Of note is the emotional 
  friction and contrast between the writings of visitor Tisdale 
  and his students, who all seem to be warmer, more direct, and 
  more violent (though Tisdale uses the language 	``unpolished") 
  in their approaches to poetry and writing than Tisdale. The 
  primary poems I wish to analyse and use in my cultural synthesis 
  are: 
  \begin{itemize}
  
    \item Poet - Raymond X. Webster
    
    \item Black Dolphin and Haiku - Harold E. Packwood  
    
    \item What Makes a Man Free? - Clarence Phillips
    
    \item The Cure - ``Jamail" Robert Simms
    
    \item 1st Page - Daniel Brown
    
    \item Remember This - Celes Tisdale
  
  \end{itemize}
  
\end{adjustwidth}

\clearpage

\nocite{*}

\printbibliography

\end{document} 
